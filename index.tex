% Options for packages loaded elsewhere
\PassOptionsToPackage{unicode}{hyperref}
\PassOptionsToPackage{hyphens}{url}
\PassOptionsToPackage{dvipsnames,svgnames,x11names}{xcolor}
%
\documentclass[
  a4paper,
  twoside]{uoe-thesis-template}

\usepackage{amsmath,amssymb}
\usepackage{iftex}
\ifPDFTeX
  \usepackage[T1]{fontenc}
  \usepackage[utf8]{inputenc}
  \usepackage{textcomp} % provide euro and other symbols
\else % if luatex or xetex
  \usepackage{unicode-math}
  \defaultfontfeatures{Scale=MatchLowercase}
  \defaultfontfeatures[\rmfamily]{Ligatures=TeX,Scale=1}
\fi
\usepackage{lmodern}
\ifPDFTeX\else  
    % xetex/luatex font selection
\fi
% Use upquote if available, for straight quotes in verbatim environments
\IfFileExists{upquote.sty}{\usepackage{upquote}}{}
\IfFileExists{microtype.sty}{% use microtype if available
  \usepackage[]{microtype}
  \UseMicrotypeSet[protrusion]{basicmath} % disable protrusion for tt fonts
}{}
\makeatletter
\@ifundefined{KOMAClassName}{% if non-KOMA class
  \IfFileExists{parskip.sty}{%
    \usepackage{parskip}
  }{% else
    \setlength{\parindent}{0pt}
    \setlength{\parskip}{6pt plus 2pt minus 1pt}}
}{% if KOMA class
  \KOMAoptions{parskip=half}}
\makeatother
\usepackage{xcolor}
\setlength{\emergencystretch}{3em} % prevent overfull lines
\setcounter{secnumdepth}{5}
% Make \paragraph and \subparagraph free-standing
\makeatletter
\ifx\paragraph\undefined\else
  \let\oldparagraph\paragraph
  \renewcommand{\paragraph}{
    \@ifstar
      \xxxParagraphStar
      \xxxParagraphNoStar
  }
  \newcommand{\xxxParagraphStar}[1]{\oldparagraph*{#1}\mbox{}}
  \newcommand{\xxxParagraphNoStar}[1]{\oldparagraph{#1}\mbox{}}
\fi
\ifx\subparagraph\undefined\else
  \let\oldsubparagraph\subparagraph
  \renewcommand{\subparagraph}{
    \@ifstar
      \xxxSubParagraphStar
      \xxxSubParagraphNoStar
  }
  \newcommand{\xxxSubParagraphStar}[1]{\oldsubparagraph*{#1}\mbox{}}
  \newcommand{\xxxSubParagraphNoStar}[1]{\oldsubparagraph{#1}\mbox{}}
\fi
\makeatother

\usepackage{color}
\usepackage{fancyvrb}
\newcommand{\VerbBar}{|}
\newcommand{\VERB}{\Verb[commandchars=\\\{\}]}
\DefineVerbatimEnvironment{Highlighting}{Verbatim}{commandchars=\\\{\}}
% Add ',fontsize=\small' for more characters per line
\usepackage{framed}
\definecolor{shadecolor}{RGB}{241,243,245}
\newenvironment{Shaded}{\begin{snugshade}}{\end{snugshade}}
\newcommand{\AlertTok}[1]{\textcolor[rgb]{0.68,0.00,0.00}{#1}}
\newcommand{\AnnotationTok}[1]{\textcolor[rgb]{0.37,0.37,0.37}{#1}}
\newcommand{\AttributeTok}[1]{\textcolor[rgb]{0.40,0.45,0.13}{#1}}
\newcommand{\BaseNTok}[1]{\textcolor[rgb]{0.68,0.00,0.00}{#1}}
\newcommand{\BuiltInTok}[1]{\textcolor[rgb]{0.00,0.23,0.31}{#1}}
\newcommand{\CharTok}[1]{\textcolor[rgb]{0.13,0.47,0.30}{#1}}
\newcommand{\CommentTok}[1]{\textcolor[rgb]{0.37,0.37,0.37}{#1}}
\newcommand{\CommentVarTok}[1]{\textcolor[rgb]{0.37,0.37,0.37}{\textit{#1}}}
\newcommand{\ConstantTok}[1]{\textcolor[rgb]{0.56,0.35,0.01}{#1}}
\newcommand{\ControlFlowTok}[1]{\textcolor[rgb]{0.00,0.23,0.31}{\textbf{#1}}}
\newcommand{\DataTypeTok}[1]{\textcolor[rgb]{0.68,0.00,0.00}{#1}}
\newcommand{\DecValTok}[1]{\textcolor[rgb]{0.68,0.00,0.00}{#1}}
\newcommand{\DocumentationTok}[1]{\textcolor[rgb]{0.37,0.37,0.37}{\textit{#1}}}
\newcommand{\ErrorTok}[1]{\textcolor[rgb]{0.68,0.00,0.00}{#1}}
\newcommand{\ExtensionTok}[1]{\textcolor[rgb]{0.00,0.23,0.31}{#1}}
\newcommand{\FloatTok}[1]{\textcolor[rgb]{0.68,0.00,0.00}{#1}}
\newcommand{\FunctionTok}[1]{\textcolor[rgb]{0.28,0.35,0.67}{#1}}
\newcommand{\ImportTok}[1]{\textcolor[rgb]{0.00,0.46,0.62}{#1}}
\newcommand{\InformationTok}[1]{\textcolor[rgb]{0.37,0.37,0.37}{#1}}
\newcommand{\KeywordTok}[1]{\textcolor[rgb]{0.00,0.23,0.31}{\textbf{#1}}}
\newcommand{\NormalTok}[1]{\textcolor[rgb]{0.00,0.23,0.31}{#1}}
\newcommand{\OperatorTok}[1]{\textcolor[rgb]{0.37,0.37,0.37}{#1}}
\newcommand{\OtherTok}[1]{\textcolor[rgb]{0.00,0.23,0.31}{#1}}
\newcommand{\PreprocessorTok}[1]{\textcolor[rgb]{0.68,0.00,0.00}{#1}}
\newcommand{\RegionMarkerTok}[1]{\textcolor[rgb]{0.00,0.23,0.31}{#1}}
\newcommand{\SpecialCharTok}[1]{\textcolor[rgb]{0.37,0.37,0.37}{#1}}
\newcommand{\SpecialStringTok}[1]{\textcolor[rgb]{0.13,0.47,0.30}{#1}}
\newcommand{\StringTok}[1]{\textcolor[rgb]{0.13,0.47,0.30}{#1}}
\newcommand{\VariableTok}[1]{\textcolor[rgb]{0.07,0.07,0.07}{#1}}
\newcommand{\VerbatimStringTok}[1]{\textcolor[rgb]{0.13,0.47,0.30}{#1}}
\newcommand{\WarningTok}[1]{\textcolor[rgb]{0.37,0.37,0.37}{\textit{#1}}}

\providecommand{\tightlist}{%
  \setlength{\itemsep}{0pt}\setlength{\parskip}{0pt}}\usepackage{longtable,booktabs,array}
\usepackage{calc} % for calculating minipage widths
% Correct order of tables after \paragraph or \subparagraph
\usepackage{etoolbox}
\makeatletter
\patchcmd\longtable{\par}{\if@noskipsec\mbox{}\fi\par}{}{}
\makeatother
% Allow footnotes in longtable head/foot
\IfFileExists{footnotehyper.sty}{\usepackage{footnotehyper}}{\usepackage{footnote}}
\makesavenoteenv{longtable}
\usepackage{graphicx}
\makeatletter
\def\maxwidth{\ifdim\Gin@nat@width>\linewidth\linewidth\else\Gin@nat@width\fi}
\def\maxheight{\ifdim\Gin@nat@height>\textheight\textheight\else\Gin@nat@height\fi}
\makeatother
% Scale images if necessary, so that they will not overflow the page
% margins by default, and it is still possible to overwrite the defaults
% using explicit options in \includegraphics[width, height, ...]{}
\setkeys{Gin}{width=\maxwidth,height=\maxheight,keepaspectratio}
% Set default figure placement to htbp
\makeatletter
\def\fps@figure{htbp}
\makeatother
% definitions for citeproc citations
\NewDocumentCommand\citeproctext{}{}
\NewDocumentCommand\citeproc{mm}{%
  \begingroup\def\citeproctext{#2}\cite{#1}\endgroup}
\makeatletter
 % allow citations to break across lines
 \let\@cite@ofmt\@firstofone
 % avoid brackets around text for \cite:
 \def\@biblabel#1{}
 \def\@cite#1#2{{#1\if@tempswa , #2\fi}}
\makeatother
\newlength{\cslhangindent}
\setlength{\cslhangindent}{1.5em}
\newlength{\csllabelwidth}
\setlength{\csllabelwidth}{3em}
\newenvironment{CSLReferences}[2] % #1 hanging-indent, #2 entry-spacing
 {\begin{list}{}{%
  \setlength{\itemindent}{0pt}
  \setlength{\leftmargin}{0pt}
  \setlength{\parsep}{0pt}
  % turn on hanging indent if param 1 is 1
  \ifodd #1
   \setlength{\leftmargin}{\cslhangindent}
   \setlength{\itemindent}{-1\cslhangindent}
  \fi
  % set entry spacing
  \setlength{\itemsep}{#2\baselineskip}}}
 {\end{list}}
\usepackage{calc}
\newcommand{\CSLBlock}[1]{\hfill\break\parbox[t]{\linewidth}{\strut\ignorespaces#1\strut}}
\newcommand{\CSLLeftMargin}[1]{\parbox[t]{\csllabelwidth}{\strut#1\strut}}
\newcommand{\CSLRightInline}[1]{\parbox[t]{\linewidth - \csllabelwidth}{\strut#1\strut}}
\newcommand{\CSLIndent}[1]{\hspace{\cslhangindent}#1}

% TODO: Add custom LaTeX header directives here

%% The following are packages that are used by the template and that you will likely need for your content. Do not change this.
\usepackage[utf8]{inputenc}
\usepackage[english]{babel}
\usepackage{CormorantGaramond}
\usepackage{hologo}
\usepackage{natbib}
\citestyle{aysep={,}}
\usepackage{graphicx}
\usepackage{float}
\usepackage{setspace}
\usepackage{url}
\usepackage[nottoc]{tocbibind}
\usepackage{enumitem}
\usepackage{subcaption}
\usepackage{siunitx}
\usepackage{lscape}
\usepackage{textgreek}

%% Under this line, you can add additional packages that you may need for typesetting your document.

%\usepackage{...}
\usepackage{listings}

\usepackage{xcolor}

%New colors defined below
\definecolor{codegreen}{rgb}{0,0.6,0}
\definecolor{codegray}{rgb}{0.5,0.5,0.5}
\definecolor{codepurple}{rgb}{0.58,0,0.82}
\definecolor{backcolour}{rgb}{0.98,0.98,0.98}

%Code listing style named "mystyle"
\lstdefinestyle{mystyle}{
  backgroundcolor=\color{backcolour}, commentstyle=\color{codegreen},
  keywordstyle=\color{magenta},
  numberstyle=\tiny\color{codegray},
  stringstyle=\color{codepurple},
  basicstyle=\ttfamily\footnotesize\tiny,
  breakatwhitespace=false,         
  breaklines=true,                 
  captionpos=b,                    
  keepspaces=true,                 
  numbers=left,                    
  numbersep=5pt,                  
  showspaces=false,                
  showstringspaces=false,
  showtabs=false,                  
  tabsize=2
}
%define Javascript language
\lstdefinelanguage{JavaScript}{
keywords={typeof, new, true, false, catch, function, return, null, catch, switch, var, if, in, while, do, else, case, break},
keywordstyle=\color{blue}\bfseries,
ndkeywords={class, export, boolean, throw, implements, import, this},
ndkeywordstyle=\color{darkgray}\bfseries,
identifierstyle=\color{black},
sensitive=false,
comment=[l]{//},
morecomment=[s]{/*}{*/},
commentstyle=\color{purple}\ttfamily,
stringstyle=\color{red}\ttfamily,
morestring=[b]',
morestring=[b]"
}
%"mystyle" code listing set
\lstset{style=mystyle}



%%%%%%%%%%%%%%%%%%%%%%%%%%%%%%%%%%%%%%%%%%%%%%%%%%%%%%%%%%%%%%%%%%%%%%%%%%%%%%%%%%%%%%%%%%%%%%%%%%%%%%%%%%%%%%%%
%% The following says that all graphics (e.g. jpg, png, etc. files) will be searched in the "Figures/" folder. 
%% Do not change it but ensure you put your figures in that folder.

\makeatletter
\@ifpackageloaded{tcolorbox}{}{\usepackage[skins,breakable]{tcolorbox}}
\@ifpackageloaded{fontawesome5}{}{\usepackage{fontawesome5}}
\definecolor{quarto-callout-color}{HTML}{909090}
\definecolor{quarto-callout-note-color}{HTML}{0758E5}
\definecolor{quarto-callout-important-color}{HTML}{CC1914}
\definecolor{quarto-callout-warning-color}{HTML}{EB9113}
\definecolor{quarto-callout-tip-color}{HTML}{00A047}
\definecolor{quarto-callout-caution-color}{HTML}{FC5300}
\definecolor{quarto-callout-color-frame}{HTML}{acacac}
\definecolor{quarto-callout-note-color-frame}{HTML}{4582ec}
\definecolor{quarto-callout-important-color-frame}{HTML}{d9534f}
\definecolor{quarto-callout-warning-color-frame}{HTML}{f0ad4e}
\definecolor{quarto-callout-tip-color-frame}{HTML}{02b875}
\definecolor{quarto-callout-caution-color-frame}{HTML}{fd7e14}
\makeatother
\makeatletter
\@ifpackageloaded{bookmark}{}{\usepackage{bookmark}}
\makeatother
\makeatletter
\@ifpackageloaded{caption}{}{\usepackage{caption}}
\AtBeginDocument{%
\ifdefined\contentsname
  \renewcommand*\contentsname{Table of contents}
\else
  \newcommand\contentsname{Table of contents}
\fi
\ifdefined\listfigurename
  \renewcommand*\listfigurename{List of Figures}
\else
  \newcommand\listfigurename{List of Figures}
\fi
\ifdefined\listtablename
  \renewcommand*\listtablename{List of Tables}
\else
  \newcommand\listtablename{List of Tables}
\fi
\ifdefined\figurename
  \renewcommand*\figurename{Figure}
\else
  \newcommand\figurename{Figure}
\fi
\ifdefined\tablename
  \renewcommand*\tablename{Table}
\else
  \newcommand\tablename{Table}
\fi
}
\@ifpackageloaded{float}{}{\usepackage{float}}
\floatstyle{ruled}
\@ifundefined{c@chapter}{\newfloat{codelisting}{h}{lop}}{\newfloat{codelisting}{h}{lop}[chapter]}
\floatname{codelisting}{Listing}
\newcommand*\listoflistings{\listof{codelisting}{List of Listings}}
\makeatother
\makeatletter
\makeatother
\makeatletter
\@ifpackageloaded{caption}{}{\usepackage{caption}}
\@ifpackageloaded{subcaption}{}{\usepackage{subcaption}}
\makeatother

\ifLuaTeX
  \usepackage{selnolig}  % disable illegal ligatures
\fi
\usepackage{bookmark}

\IfFileExists{xurl.sty}{\usepackage{xurl}}{} % add URL line breaks if available
\urlstyle{same} % disable monospaced font for URLs
\hypersetup{
  pdftitle={Deep models for shallow waters},
  pdfauthor={James Harding},
  pdfkeywords={template, demo},
  colorlinks=true,
  linkcolor={blue},
  filecolor={Maroon},
  citecolor={Blue},
  urlcolor={Blue},
  pdfcreator={LaTeX via pandoc}}


%%%%% THESIS / TITLE PAGE INFORMATION
% Everybody needs to complete the following:
\title{Deep models for shallow waters}
\author{James Harding}
\college{School for Infrastructure and Environment}

% Master's candidates who require the alternate title page (with candidate number and word count)
% must also un-comment and complete the following three lines:
%\masterssubmissiontrue
%\candidateno{933516}
%\wordcount{28,815}

% Uncomment the following line if your degree also includes exams (eg most masters):
%\renewcommand{\submittedtext}{Submitted in partial completion of the}
% Your full degree name.  (But remember that DPhils aren't "in" anything.  They're just DPhils.)
\degree{Doctor of Philosophy}
% Term and year of submission, or date if your board requires (eg most masters)
\degreedate{2023}
\begin{document}
%%%%% CHOOSE YOUR LINE SPACING HERE
% This is the official option.  Use it for your submission copy and library copy:
\setlength{\textbaselineskip}{22pt plus2pt}
% This is closer spacing (about 1.5-spaced) that you might prefer for your personal copies:
%\setlength{\textbaselineskip}{18pt plus2pt minus1pt}

% You can set the spacing here for the roman-numbered pages (acknowledgements, table of contents, etc.)
\setlength{\frontmatterbaselineskip}{17pt plus1pt minus1pt}

% Leave this line alone; it gets things started for the real document.
\setlength{\baselineskip}{\textbaselineskip}

%%%%% CHOOSE YOUR SECTION NUMBERING DEPTH HERE
% You have two choices.  First, how far down are sections numbered?  (Below that, they're named but
% don't get numbers.)  Second, what level of section appears in the table of contents?  These don't have
% to match: you can have numbered sections that don't show up in the ToC, or unnumbered sections that
% do.  Throughout, 0 = chapter; 1 = section; 2 = subsection; 3 = subsubsection, 4 = paragraph...

% The level that gets a number:
\setcounter{secnumdepth}{2}
% The level that shows up in the ToC:
\setcounter{tocdepth}{2}


%%%%% ABSTRACT SEPARATE
% This is used to create the separate, one-page abstract that you are required to hand into the Exam
% Schools.  You can comment it out to generate a PDF for printing or whatnot.
\begin{abstractseparate}
	\input{frontmatter/03-abstract.txt} % Create an abstract.tex file in the 'text' folder for your abstract.
\end{abstractseparate}

\newpage
\thispagestyle{empty}
\null
\newpage
% JEM: Pages are roman numbered from here, though page numbers are invisible until ToC.  This is in
% keeping with most typesetting conventions.
\pagenumbering{roman}

\begin{romanpages}

% Title page is created here
\maketitle

%%%%% DEDICATION -- If you'd like one, un-comment the following.

\begin{dedication}
    \input{frontmatter/01-dedication.txt}
\end{dedication}

%%%%% Declaration -- Nothing to do here except comment out if you don't want it.

\begin{declaration}
\addcontentsline{toc}{chapter}{Declaration}
	\input{frontmatter/02-declaration.txt}
    \vspace{0.5em}
    \begin{flushright}
      \includegraphics[width=4 cm]{frontmatter/figs/signature.jpg}
    \end{flushright}
    \vspace{0.5em}
    \begin{flushright}
      {\large \textit{James Harding}}
    \end{flushright}
    \vspace{0.5em}
    \noindent Word Count: tbc
    \vfill
    \noindent This thesis was conducted under the supervision of Dr
Dre..
    \vspace{2cm}
\end{declaration}

%%%%% ABSTRACT -- Nothing to do here except comment out if you don't want it.

\begin{abstract}
\addcontentsline{toc}{chapter}{Abstract}
	\input{frontmatter/03-abstract.txt}
\end{abstract}

%%%%% ACKNOWLEDGEMENTS -- Nothing to do here except comment out if you don't want it.

\begin{acknowledgements}
\addcontentsline{toc}{chapter}{Acknowledgements}
 	\input{frontmatter/04-acknowledgements.txt}
\end{acknowledgements}

% This aligns the bottom of the text of each page.  It generally makes things look better.
\flushbottom

% This is where the whole-document ToC appears except for Glossary:
\tableofcontents

\listoffigures

\listoftables

\end{romanpages}

\flushbottom
\bookmarksetup{startatroot}

\chapter*{Glossary}\label{glossary}
\addcontentsline{toc}{chapter}{Glossary}

\markboth{Glossary}{Glossary}

\section*{Abbreviations}\label{abbreviations}
\addcontentsline{toc}{section}{Abbreviations}

\markright{Abbreviations}

\begin{description}
\tightlist
\item[AI]
Artificial Intelligence
\item[CNN]
Convolutional Neural Network
\item[GAN]
Generative Adversarial Network
\item[CNN]
Convolutional Neural Network
\item[GAN]
Generative Adversarial Network
\item[GRU]
Gated Recurrent Unit
\item[LSTM]
Long Short-Term Memory
\item[MLP]
Multi-Layer Perceptron
\item[RBM]
Restricted Boltzmann Machine
\item[TCN]
Temporal Convolutional Network
\item[VAE]
Variational Autoencoder
\item[WD]
Wasserstein-Disrtance
\item[cVAE]
Conditional Variational Autoencoder
\item[GAIN]
Generative Adversarial Imputation Networks
\item[RMS(L)E]
Root Mean Squared Error
\item[G-means]
G-means
\item[NLL]
Negative Log-Likelihood
\item[MCC]
Matthews Correlation Coefficient
\end{description}

\section*{Parameters}\label{parameters}
\addcontentsline{toc}{section}{Parameters}

\markright{Parameters}

\begin{description}
\tightlist
\item[\(\text{Fe}\)]
Iron (mg/L)
\item[\(\text{Al}\)]
Aluminum (mg/L)
\end{description}

\section*{Nomenclature}\label{nomenclature}
\addcontentsline{toc}{section}{Nomenclature}

\markright{Nomenclature}

\begin{description}
\tightlist
\item[\(J\)]
cost/loss function
\item[\(\nabla\)]
derivative of cost w.r.t. model parameters
\item[\(w\)]
model parameters
\item[\(b\)]
model parameters
\item[\(x\)]
input data
\item[\(y\)]
output data
\item[\(\mu\)]
mean
\item[\(\sigma^2\)]
variance
\end{description}

\section*{Terminology}\label{terminology}
\addcontentsline{toc}{section}{Terminology}

\markright{Terminology}

\begin{description}
\tightlist
\item[gradient]
derivative of cost w.r.t. model parameters
\item[optimizer]
algorithm used to minimize the cost function
\item[batch]
subset of training data used in one iteration
\item[epoch]
one pass through the entire training dataset
\item[overfitting]
model is memorizing the training data and not generalizing well to new
data
\item[underfitting]
model is not able to learn the underlying patterns in the data
\item[regularization]
optimisation by adding a penalty term to the loss function
\item[dropout]
optimise by randomly dropping out neurons during training
\end{description}

\bookmarksetup{startatroot}

\chapter*{Preface}\label{preface}
\addcontentsline{toc}{chapter}{Preface}

\markboth{Preface}{Preface}

\setcounter{page}{1}
\renewcommand{\thepage}{\arabic{page}}

\begin{tcolorbox}[enhanced jigsaw, opacitybacktitle=0.6, colbacktitle=quarto-callout-note-color!10!white, colframe=quarto-callout-note-color-frame, leftrule=.75mm, bottomtitle=1mm, toptitle=1mm, toprule=.15mm, colback=white, left=2mm, titlerule=0mm, opacityback=0, rightrule=.15mm, arc=.35mm, title=\textcolor{quarto-callout-note-color}{\faInfo}\hspace{0.5em}{Note}, bottomrule=.15mm, coltitle=black, breakable]

Leave or move the latex snippet above to adjust when arabic page numbers
begin. That is also the format for injecting various other code.

\end{tcolorbox}

\section*{Background}\label{sec-intro}
\addcontentsline{toc}{section}{Background}

\markright{Background}

Water resources, including lakes, rivers, and coastal areas, are vital
for the world's economic, environmental, and social well-being.
Virtutibus occurreret ab perfectius ob majestatis ex ei. Sex ego uno
locum lor clara istas. Haustam organis veritas pro sed. Extensarum
imperfecta vox per propugnent cap utilitatis ero. Tot callidus venturum
pictores subducam appareat hae ubi. Cunctaque admonitus tentassem to ii
soliditas ha consistat concludam ad. Est ritas sae aut istam paulo reges
terea serie fas.

Animalium cui duo credendas potentiam nuperrime fruebatur. Ne si nostrae
mallent possunt ea extendo. Quo dici nam has ulla eas idem quum. Dormiam
conabor eo videmus prudens allatis ea in. Age effectibus argumentis
constanter uno occasionem. Nudi dura ob ac suas aspi et de fere. Si
prius magna ab in fides co. Lor hoc statuere callidum interire rei imo.

Ausi bere ut erit adeo enim an suae. Petebat proprie suo methodo ope sui
admodum. Sane in tria nunc ne fore loco. Cum eam ullo nota duas. Ut ii
nemine augeri majora. Ad numquam naturas deumque intueri limites si.
Persuaderi eam imaginando vox res procederet intellectu imo.

\section*{Problem Domain}\label{problem-domain}
\addcontentsline{toc}{section}{Problem Domain}

\markright{Problem Domain}

Animalium cui duo credendas potentiam nuperrime fruebatur. Ne si nostrae
mallent possunt ea extendo. Quo dici nam has ulla eas idem quum. Dormiam
conabor eo videmus prudens allatis ea in. Age effectibus argumentis
constanter uno occasionem. Nudi dura ob ac suas aspi et de fere. Si
prius magna ab in fides co. Lor hoc statuere callidum interire rei imo.

\begin{tcolorbox}[enhanced jigsaw, opacitybacktitle=0.6, colbacktitle=quarto-callout-note-color!10!white, colframe=quarto-callout-note-color-frame, leftrule=.75mm, bottomtitle=1mm, toptitle=1mm, toprule=.15mm, colback=white, left=2mm, titlerule=0mm, opacityback=0, rightrule=.15mm, arc=.35mm, title=\textcolor{quarto-callout-note-color}{\faInfo}\hspace{0.5em}{Note}, bottomrule=.15mm, coltitle=black, breakable]

Lists

\end{tcolorbox}

\begin{itemize}
\tightlist
\item
  Complexity: this is how you do lists in markdown.
\item
  Inconsistency: or many other straight ovious ways (1. , - * etc).
\item
  Limited parameters: Si prius magna ab in fides co. Lor hoc statuere
  callidum interire rei imo.
\item
  Integration with traditional Si prius magna ab in fides co. Lor hoc
  statuere callidum interire rei imo.
\item
  Uncertainties: Si prius magna ab in fides co. Lor hoc statuere
  callidum interire rei imo.
\item
  White-box models: Si prius magna ab in fides co. Lor hoc statuere
  callidum interire rei imo.
\end{itemize}

Si prius magna ab in fides co. Lor hoc statuere callidum interire rei
imo.

\section*{Motivation}\label{motivation}
\addcontentsline{toc}{section}{Motivation}

\markright{Motivation}

Mentemque persuadet ei opportune de aggredior proponere. Imaginabar
objectioni indefinite ne ab propositio. Ex vera iste quam mo mihi fere
post. Rogo meae imo bono aër vidi non sint. In refutent ea utrimque
extensio re tractare ex rationem. Dixi omni quas re se poni is eram. Una
mundo tangi sub tam capax porro vel talia sonum. Dulcedinem praecipuum
vox desiderant hic hauriantur sed tractandae. Realiter reperiri collecta
at an in. Quodque ne im ab ex hominem usitata apertum. Tum judicium sua
age automata eos quicquam. Si confirmari persuadeor praemissae
satyriscos cogitantem et et. Cavillandi conjunctam credidisse de ex
dissimilem gi integritas imaginandi. Examino plausum sub attendo hos.

\section*{Research Aims and
Objectives}\label{research-aims-and-objectives}
\addcontentsline{toc}{section}{Research Aims and Objectives}

\markright{Research Aims and Objectives}

\begin{tcolorbox}[enhanced jigsaw, opacitybacktitle=0.6, colbacktitle=quarto-callout-note-color!10!white, colframe=quarto-callout-note-color-frame, leftrule=.75mm, bottomtitle=1mm, toptitle=1mm, toprule=.15mm, colback=white, left=2mm, titlerule=0mm, opacityback=0, rightrule=.15mm, arc=.35mm, title=\textcolor{quarto-callout-note-color}{\faInfo}\hspace{0.5em}{Note}, bottomrule=.15mm, coltitle=black, breakable]

More lists

\end{tcolorbox}

\begin{enumerate}
\def\labelenumi{\arabic{enumi}.}
\tightlist
\item
  Heres
\item
  A numbered list
\item
  much quicker to type!
\end{enumerate}

\section*{Contribution and
Beneficiaries}\label{contribution-and-beneficiaries}
\addcontentsline{toc}{section}{Contribution and Beneficiaries}

\markright{Contribution and Beneficiaries}

Dixi omni quas re se poni is eram. Una mundo tangi sub tam capax porro
vel talia sonum. Dulcedinem praecipuum vox desiderant hic hauriantur sed
tractandae. Realiter reperiri collecta at an in. Quodque ne im ab ex
hominem usitata apertum. Tum judicium sua age automata eos quicquam. Si
confirmari persuadeor praemissae satyriscos cogitantem et et. Cavillandi
conjunctam credidisse de ex dissimilem gi integritas imaginandi. Examino
plausum sub attendo hos.

\section*{Research Questions}\label{research-questions}
\addcontentsline{toc}{section}{Research Questions}

\markright{Research Questions}

\begin{tcolorbox}[enhanced jigsaw, opacitybacktitle=0.6, colbacktitle=quarto-callout-note-color!10!white, colframe=quarto-callout-note-color-frame, leftrule=.75mm, bottomtitle=1mm, toptitle=1mm, toprule=.15mm, colback=white, left=2mm, titlerule=0mm, opacityback=0, rightrule=.15mm, arc=.35mm, title=\textcolor{quarto-callout-note-color}{\faInfo}\hspace{0.5em}{Note}, bottomrule=.15mm, coltitle=black, breakable]

Enough lists:

\end{tcolorbox}

\begin{itemize}
\tightlist
\item
  other kinds of lists?
\item
  Jam hae motarum mem luminis utilius sum. Superare tractare re ad
  videntur ha is mac
\item
  Jam hae motarum mem luminis utilius sum. Superare tractare re ad
  videntur ha is mac
\end{itemize}

\section*{Problem Redefined}\label{problem-redefined}
\addcontentsline{toc}{section}{Problem Redefined}

\markright{Problem Redefined}

Smarter now Rei discrepant probabiles distribuam nec extensarum
designabam. Admisi nec sacras mea cupere certum uti tur. De co is ad
autem pedem fidem sciri tango. Praefatio de similibus evidenter de
animalium. Sim sacras causas figere pro istius rom. Scopo uno non nolim
omnem supra istis mente.

\section*{Thesis Structure}\label{thesis-structure}
\addcontentsline{toc}{section}{Thesis Structure}

\markright{Thesis Structure}

This thesis is divided into the following chapters:

\begin{itemize}
\tightlist
\item
  Introduction
\item
  Literature Review
\item
  Materials and Methods
\item
  Deep Generative Models
\item
  Results
\item
  Conclusion
\item
  References
\item
  Appendix
\end{itemize}

\section*{Data Overview}\label{data-overview}
\addcontentsline{toc}{section}{Data Overview}

\markright{Data Overview}

Tactio sequor audita primae mentis sex cap. At rerumque credamus ex
ostendam et timerent exsurgit ac. Manifestum perficitur perspicuum ac
continetur ut to si objectioni. Audita fuerit du videam quodam ab de
figere vi. Nocturna superque at exemplum im cogitans de. Ut an
credidique diversorum appellatur si. Sum res agam cito est fide.

\bookmarksetup{startatroot}

\chapter{Literature Review}\label{literature-review}

\section{Introduction}\label{introduction}

\begin{tcolorbox}[enhanced jigsaw, opacitybacktitle=0.6, colbacktitle=quarto-callout-note-color!10!white, colframe=quarto-callout-note-color-frame, leftrule=.75mm, bottomtitle=1mm, toptitle=1mm, toprule=.15mm, colback=white, left=2mm, titlerule=0mm, opacityback=0, rightrule=.15mm, arc=.35mm, title=\textcolor{quarto-callout-note-color}{\faInfo}\hspace{0.5em}{Note}, bottomrule=.15mm, coltitle=black, breakable]

figures and cross-referencing. The figure is generated in a jupyterlab
.ipynb notebook here quarto-phd-template/chapter-01/notebooks. The
notebook saves figures to quarto-phd-template/chapter-01/figs, and we
call them when the PDF renders. Reproducible, easy to manage plots!

\end{tcolorbox}

This Thesis will consider three areas of literature review that
encompass the work, shown diagrammatically below in
Figure~\ref{fig-venn}:

\begin{figure}

\centering{

\includegraphics{chapter-01/figs/thiswork_venn_diagram.pdf}

}

\caption{\label{fig-venn}Venn-diagram of research fields encompassing
this work.}

\end{figure}%

Fuerint certius dormire duratio expirat mea has agendis. Sequeretur et
praecipuus recensenda du gi pensitatis ei intelligam. Est externarum sit
scripturas praemissae. Nulla demus has rebus timet sui mecum certe.
Facultate affirmare is ac priusquam tribuebam potentiam et ex continent.
Unam pati suae vice hos luce addi dem. Meos ipsa atra vi unam in tale.
Reges istam mundo spero ad at ha nolle.

\begin{tcolorbox}[enhanced jigsaw, opacitybacktitle=0.6, colbacktitle=quarto-callout-note-color!10!white, colframe=quarto-callout-note-color-frame, leftrule=.75mm, bottomtitle=1mm, toptitle=1mm, toprule=.15mm, colback=white, left=2mm, titlerule=0mm, opacityback=0, rightrule=.15mm, arc=.35mm, title=\textcolor{quarto-callout-note-color}{\faInfo}\hspace{0.5em}{Note}, bottomrule=.15mm, coltitle=black, breakable]

We call the other two plots with some publication stats using the below.
See quarto documentation for many ways to plot multiple images. Make
sure to leave a line spacing between code as below. The last \{\}
declares sizing, cross-ref tegs and caption locations.

\end{tcolorbox}

\begin{figure}

\begin{minipage}{0.50\linewidth}

\centering{

\includegraphics{chapter-01/figs/publications_bar_chart.png}

}

\subcaption{\label{fig-totpub}Total Publications 2014-2023}

\end{minipage}%
%
\begin{minipage}{0.50\linewidth}

\centering{

\includegraphics{chapter-01/figs/relative_change_bar_chart.png}

}

\subcaption{\label{fig-pubchange}Relative 5-Year Change 2017-2022}

\end{minipage}%

\caption{\label{fig-elephants}Analytical view of prior work in some
research fields per Figure~\ref{fig-venn}, including their
intersections. @ is how all references occur from .bibs to sections etc.
But you can use latex etc too. P.S. This is the figure caption.}

\end{figure}%

\begin{tcolorbox}[enhanced jigsaw, opacitybacktitle=0.6, colbacktitle=quarto-callout-note-color!10!white, colframe=quarto-callout-note-color-frame, leftrule=.75mm, bottomtitle=1mm, toptitle=1mm, toprule=.15mm, colback=white, left=2mm, titlerule=0mm, opacityback=0, rightrule=.15mm, arc=.35mm, title=\textcolor{quarto-callout-note-color}{\faInfo}\hspace{0.5em}{Note}, bottomrule=.15mm, coltitle=black, breakable]

Note how much easier it is to read write than latex - .md (markdown is
preferred by web developers for this reason.

\end{tcolorbox}

Animalium cui duo credendas potentiam nuperrime fruebatur. Ne si nostrae
mallent possunt ea extendo. Quo dici nam has ulla eas idem quum. Dormiam
conabor eo videmus prudens allatis ea in. Age effectibus argumentis
constanter uno occasionem. Nudi dura ob ac suas aspi et de fere. Si
prius magna ab in fides co. Lor hoc statuere callidum interire rei imo.
Animalium cui duo credendas potentiam nuperrime fruebatur. Ne si nostrae
mallent possunt ea extendo. Quo dici nam has ulla eas idem quum. Dormiam
conabor eo videmus prudens allatis ea in. Age effectibus argumentis
constanter uno occasionem. Nudi dura ob ac suas aspi et de fere. Si
prius magna ab in fides co. Lor hoc statuere callidum interire rei imo.
Sane quos unde esto sed una est bere. Funditus co is formemus converto
id. Equidem quidnam at si mutetur frigida innatas. Cur vitari angeli iii
haberi rea. Luce ei quem scio vita ne apud. Is dicunt patere vi putavi
ne. Percipitur propositio co in cognitione id perciperem et alloquendo.
Rem idemque sex exigere credidi scripto. Mentemque persuadet ei
opportune de aggredior proponere. Imaginabar objectioni indefinite ne ab
propositio. Ex vera iste quam mo mihi fere post. Rogo meae imo bono aër
vidi non sint. In refutent ea utrimque extensio re tractare ex rationem.
Dixi omni quas re se poni is eram. Una mundo tangi sub tam capax porro
vel talia sonum. Dulcedinem praecipuum vox desiderant hic hauriantur sed
tractandae. Realiter reperiri collecta at an in. Quodque ne im ab ex
hominem usitata apertum. Tum judicium sua age automata eos quicquam. Si
confirmari persuadeor praemissae satyriscos cogitantem et et. Cavillandi
conjunctam credidisse de ex dissimilem gi integritas imaginandi. Examino
plausum sub attendo hos. Jam hae motarum mem luminis utilius sum.
Superare tractare re ad videntur ha is machinae cogitans partibus.
Quisquis revocari eo quidquam ut patiatur imaginor. Vi immortalem de si
cucurbitas perfectior desiderant. Et at concedam delapsus mutuatis
importat rogassem ad. Sap obversari mei conjectus contineri pro
distingui vix excludere. Sensibus quanquam et ac imagines infinite
statuere judicium. Vetus versa ita negat mea nudam qua hoc foret illae.
Rei discrepant probabiles distribuam nec extensarum designabam. Admisi
nec sacras mea cupere certum uti tur. De co is ad autem pedem fidem
sciri tango. Praefatio de similibus evidenter de animalium. Sim sacras
causas figere pro istius rom. Scopo uno non nolim omnem supra istis
mente. Fuerint certius dormire duratio expirat mea has agendis.
Sequeretur et praecipuus recensenda du gi pensitatis ei intelligam. Est
externarum sit scripturas praemissae. Nulla demus has rebus timet sui
mecum certe. Facultate affirmare is ac priusquam tribuebam potentiam et
ex continent. Unam pati suae vice hos luce addi dem. Meos ipsa atra vi
unam in tale. Reges istam mundo spero ad at ha nolle.

\section{Water}\label{sec-lit-wq}

\subsection{Crisis}\label{crisis}

Mentemque persuadet ei opportune de aggredior proponere. Imaginabar
objectioni indefinite ne ab propositio. Ex vera iste quam mo mihi fere
post. Rogo meae imo bono aër vidi non sint. In refutent ea utrimque
extensio re tractare ex rationem. Dixi omni quas re se poni is eram. Una
mundo tangi sub tam capax porro vel talia sonum. Dulcedinem praecipuum
vox desiderant hic hauriantur sed tractandae.

Realiter reperiri collecta at an in. Quodque ne im ab ex hominem usitata
apertum. Tum judicium sua age automata eos quicquam. Si confirmari
persuadeor praemissae satyriscos cogitantem et et. Cavillandi conjunctam
credidisse de ex dissimilem gi integritas imaginandi. Examino plausum
sub attendo hos.

\begin{tcolorbox}[enhanced jigsaw, opacitybacktitle=0.6, colbacktitle=quarto-callout-note-color!10!white, colframe=quarto-callout-note-color-frame, leftrule=.75mm, bottomtitle=1mm, toptitle=1mm, toprule=.15mm, colback=white, left=2mm, titlerule=0mm, opacityback=0, rightrule=.15mm, arc=.35mm, title=\textcolor{quarto-callout-note-color}{\faInfo}\hspace{0.5em}{Note}, bottomrule=.15mm, coltitle=black, breakable]

Above demonstrates defining leadig directories for figs and tables.
Below shows you can continue to use latex if you wish. Or mix and
match..

\end{tcolorbox}

\section{Section - Remote-sensing by
markdown}\label{section---remote-sensing-by-markdown}

\section{Section - Remote-sensing by latex}

Sane quos unde esto sed una est bere. Funditus co is formemus converto
id. Equidem quidnam at si mutetur frigida innatas. Cur vitari angeli iii
haberi rea. Luce ei quem scio vita ne apud. Is dicunt patere vi putavi
ne. Percipitur propositio co in cognitione id perciperem et alloquendo.
Rem idemque sex exigere credidi scripto. Mentemque persuadet ei
opportune de aggredior proponere. Imaginabar objectioni indefinite ne ab
propositio. Ex vera iste quam mo mihi fere post. Rogo meae imo bono aër
vidi non sint. In refutent ea utrimque extensio re tractare ex rationem.
Dixi omni quas re se poni is eram. Una mundo tangi sub tam capax porro
vel talia sonum. Dulcedinem praecipuum vox desiderant hic hauriantur sed
tractandae. Realiter reperiri collecta at an in. Quodque ne im ab ex
hominem usitata apertum. Tum judicium sua age automata eos quicquam. Si
confirmari persuadeor praemissae satyriscos cogitantem et et. Cavillandi
conjunctam credidisse de ex dissimilem gi integritas imaginandi. Examino
plausum sub attendo hos. Jam hae motarum mem luminis utilius sum.
Superare tractare re ad videntur ha is machinae cogitans partibus.
Quisquis revocari eo quidquam ut patiatur imaginor. Vi immortalem de si
cucurbitas perfectior desiderant. Et at concedam delapsus mutuatis
importat rogassem ad. Sap obversari mei conjectus contineri pro
distingui vix excludere. Sensibus quanquam et ac imagines infinite
statuere judicium. Vetus versa ita negat mea nudam qua hoc foret illae.
Rei discrepant probabiles distribuam nec extensarum designabam. Admisi
nec sacras mea cupere certum uti tur. De co is ad autem pedem fidem
sciri tango. Praefatio de similibus evidenter de animalium. Sim sacras
causas figere pro istius rom. Scopo uno non nolim omnem supra istis
mente. Fuerint certius dormire duratio expirat mea has agendis.
Sequeretur et praecipuus recensenda du gi pensitatis ei intelligam. Est
externarum sit scripturas praemissae. Nulla demus has rebus timet sui
mecum certe. Facultate affirmare is ac priusquam tribuebam potentiam et
ex continent. Unam pati suae vice hos luce addi dem. Meos ipsa atra vi
unam in tale. Reges istam mundo spero ad at ha nolle.

\begin{table}[]
\centering
\resizebox{\columnwidth}{!}{%
\begin{tabular}{lll}
\hline
\textbf{Parameter}                 & \textbf{Spaceborne}                                      & \textbf{Airborne}                 \\ \hline
Time of overpass                   & Mostly fixed                                             & Flexible                          \\
Spatial resolution &
  GSD \textless 0.5 m panchromatic / \textless 2km multi-band &
  Ground Sampling Distance (GSD) \textless 5 m \\
Spectral resolution                & Panchromatic to multispectral, some hyper-spectral       & Panchromatic to hyperspectral     \\
Temporal resolution & Days                                                     & Minutes                           \\
Calibration                        & Pre-calibration before launch, onboard characterisation & Before launch + possible on-board \\
Cost                               & Free to $\sim$\$2–10k per scene (High res)               & \$350 per square mile             \\
Stability                          & High                                                     & Low, due to turbulence            \\
Swath width &
  High - 2500 km to full hemisphere  &
  \begin{tabular}[c]{@{}l@{}}Small (up to 10 km per\\ flight line)\end{tabular} \\
Interpretation approaches          & Empirical-and semi-empirical mostly                      & Empirical and analytical          \\
Image processing complexity    & Less complex vs. hyper-spectral sensors                  & More complex, specific skills     \\
Constraints                        & Fixed spatiotemporal resolution / weather constraints    & Coverage schedule is flexible     \\
Geographic coverage areas          & Local, regional, and global                              & Local and regional                \\ \hline
                                   &                                                          &                                  
\end{tabular}%
}
\caption{Comparison between spaceborne and airborne sensors.}

\end{table}

\begin{tcolorbox}[enhanced jigsaw, opacitybacktitle=0.6, colbacktitle=quarto-callout-note-color!10!white, colframe=quarto-callout-note-color-frame, leftrule=.75mm, bottomtitle=1mm, toptitle=1mm, toprule=.15mm, colback=white, left=2mm, titlerule=0mm, opacityback=0, rightrule=.15mm, arc=.35mm, title=\textcolor{quarto-callout-note-color}{\faInfo}\hspace{0.5em}{Note}, bottomrule=.15mm, coltitle=black, breakable]

the above code just inserts a table written in another file to keep this
area clean. You are also able to make tables using markdown however.
Check this self-adjusting table out, note how positinoing impacts table
format and justification:

\end{tcolorbox}

\begin{longtable}[]{@{}llrc@{}}
\caption{Demonstration of pipe table syntax}\tabularnewline
\toprule\noalign{}
Default & Left & Right & Center \\
\midrule\noalign{}
\endfirsthead
\toprule\noalign{}
Default & Left & Right & Center \\
\midrule\noalign{}
\endhead
\bottomrule\noalign{}
\endlastfoot
12 & 12 & 12 & 12 \\
123 & 123 & 123 & 123 \\
1 & 1 & 1 & 1 \\
\end{longtable}

\newpage{}

\begin{tcolorbox}[enhanced jigsaw, opacitybacktitle=0.6, colbacktitle=quarto-callout-note-color!10!white, colframe=quarto-callout-note-color-frame, leftrule=.75mm, bottomtitle=1mm, toptitle=1mm, toprule=.15mm, colback=white, left=2mm, titlerule=0mm, opacityback=0, rightrule=.15mm, arc=.35mm, title=\textcolor{quarto-callout-note-color}{\faInfo}\hspace{0.5em}{Note}, bottomrule=.15mm, coltitle=black, breakable]

Or you can insert csv files directly, and navigate through them or
select variables. and of course you can still do things the latex way,
such as for images:

\end{tcolorbox}

\begin{figure}[h]
    \centering
    \includegraphics[width=1.\textwidth]{chapter-01/figs/l7l8s2bands.png}
    \caption{Spectral response of Landsat 7, Landsat 8 and Sentinel 2 over the electromagnetic spectrum.}
    \label{l7l8s2bands}
\end{figure}

\begin{tcolorbox}[enhanced jigsaw, opacitybacktitle=0.6, colbacktitle=quarto-callout-note-color!10!white, colframe=quarto-callout-note-color-frame, leftrule=.75mm, bottomtitle=1mm, toptitle=1mm, toprule=.15mm, colback=white, left=2mm, titlerule=0mm, opacityback=0, rightrule=.15mm, arc=.35mm, title=\textcolor{quarto-callout-note-color}{\faInfo}\hspace{0.5em}{Note}, bottomrule=.15mm, coltitle=black, breakable]

many ways to do equations too:

\end{tcolorbox}

Some text

\begin{equation}
\alpha_i=\frac{\exp \left(z_i^\alpha\right)}{\sum_{j=1}^M \exp \left(z_j^\alpha\right)}
\end{equation}
The variances $\sigma(x)$ represent scale parameters which cannot be less than zero, so we constrain its output using the exponential activation:
\begin{equation}
\sigma_i=\exp \left(z_i^\sigma\right)
\end{equation}
While no activation is applied to the $\mu(x)$ value:
\begin{equation}
\mu_{i k}=z_{i k}^\mu
\end{equation}
we can form a cost function using the equation for likelihood (of some distribution fitting the given observations $t$). Further numerical stability is ensured by taking the log-likelihood. As it is typical to minimise a function when training networks, we take the negative of the likelihood:  
\begin{equation}
E=\sum_q E^q
\end{equation}

\bookmarksetup{startatroot}

\chapter{}\label{section}

\#Materials and Methods

\begin{tcolorbox}[enhanced jigsaw, opacitybacktitle=0.6, colbacktitle=quarto-callout-note-color!10!white, colframe=quarto-callout-note-color-frame, leftrule=.75mm, bottomtitle=1mm, toptitle=1mm, toprule=.15mm, colback=white, left=2mm, titlerule=0mm, opacityback=0, rightrule=.15mm, arc=.35mm, title=\textcolor{quarto-callout-note-color}{\faInfo}\hspace{0.5em}{Note}, bottomrule=.15mm, coltitle=black, breakable]

Labels similarly work for latex as above. This chapter just shows you
can do everything in one file, or in many, slice it however works best
for your situation.

\end{tcolorbox}

Sane quos unde esto sed una est bere. Funditus co is formemus converto
id. Equidem quidnam at si mutetur frigida innatas. Cur vitari angeli iii
haberi rea. Luce ei quem scio vita ne apud. Is dicunt patere vi putavi
ne. Percipitur propositio co in cognitione id perciperem et alloquendo.
Rem idemque sex exigere credidi scripto. Mentemque persuadet ei
opportune de aggredior proponere. Imaginabar objectioni indefinite ne ab
propositio. Ex vera iste quam mo mihi fere post. Rogo meae imo bono aër
vidi non sint. In refutent ea utrimque extensio re tractare ex rationem.
Dixi omni quas re se poni is eram. Una mundo tangi sub tam capax porro
vel talia sonum. Dulcedinem praecipuum vox desiderant hic hauriantur sed
tractandae. Realiter reperiri collecta at an in. Quodque ne im ab ex
hominem usitata apertum. Tum judicium sua age automata eos quicquam. Si
confirmari persuadeor praemissae satyriscos cogitantem et et. Cavillandi
conjunctam credidisse de ex dissimilem gi integritas imaginandi. Examino
plausum sub attendo hos. Jam hae motarum mem luminis utilius sum.
Superare tractare re ad videntur ha is machinae cogitans partibus.
Quisquis revocari eo quidquam ut patiatur imaginor. Vi immortalem de si
cucurbitas perfectior desiderant. Et at concedam delapsus mutuatis
importat rogassem ad. Sap obversari mei conjectus contineri pro
distingui vix excludere. Sensibus quanquam et ac imagines infinite
statuere judicium. Vetus versa ita negat mea nudam qua hoc foret illae.
Rei discrepant probabiles distribuam nec extensarum designabam. Admisi
nec sacras mea cupere certum uti tur. De co is ad autem pedem fidem
sciri tango. Praefatio de similibus evidenter de animalium. Sim sacras
causas figere pro istius rom. Scopo uno non nolim omnem supra istis
mente. Fuerint certius dormire duratio expirat mea has agendis.
Sequeretur et praecipuus recensenda du gi pensitatis ei intelligam. Est
externarum sit scripturas praemissae. Nulla demus has rebus timet sui
mecum certe. Facultate affirmare is ac priusquam tribuebam potentiam et
ex continent. Unam pati suae vice hos luce addi dem. Meos ipsa atra vi
unam in tale. Reges istam mundo spero ad at ha nolle.

\bookmarksetup{startatroot}

\chapter{Functions}\label{functions}

\begin{enumerate}
\def\labelenumi{\arabic{enumi}.}
\item
  This is the folder structure generated when you call the cell in
  quarto-phd-template/create-chapter.ipynb. It determines your next
  chapter from the repo.
\item
  The ascii-repo.ipynb creates diagrams of the current folder structure,
  the first to one level the latter all the way. The structure is
  flexible, except index.qmd and \_quarto.yml must stay. References to
  impotant environmental legislation is also siple to do, and much the
  same as latex EU-WFD (n.d.).
\item
  Change quarto-phd-template/frontmatter/figs/signature.jpg for your own
  to apear at the declaration. Same for the uni crest in the same
  folder. index.qmd contains the Glossary - in simple Markdown. you can
  wrap that into invisible tables if you wish more control.
\end{enumerate}

\begin{Shaded}
\begin{Highlighting}[]
\NormalTok{├── .quarto/}
\NormalTok{│   ├── cites/}
\NormalTok{│   ├── idx/}
\NormalTok{│   └── xref/}
\NormalTok{├── \_book/}
\NormalTok{│   └── uoe{-}thesis.pdf}
\NormalTok{├── \_configs/}
\NormalTok{│   └── matplotlibrc.txt}
\NormalTok{├── chapter{-}00/}
\NormalTok{│   └── \_introduction.qmd}
\NormalTok{├── chapter{-}01/}
\NormalTok{│   ├── 01{-}intro.qmd}
\NormalTok{│   ├── 02{-}wq.qmd}
\NormalTok{│   ├── 03{-}eo.qmd}
\NormalTok{│   ├── 05{-}ai.qmd}
\NormalTok{│   ├── \_literature{-}review.qmd}
\NormalTok{│   ├── figs/}
\NormalTok{│   ├── images/}
\NormalTok{│   ├── litreview.qmd}
\NormalTok{│   ├── notebooks/}
\NormalTok{│   └── tables/}
\NormalTok{├── chapter{-}02/}
\NormalTok{│   └── \_materials\&methods.qmd}
\NormalTok{├── chapter{-}03/}
\NormalTok{│   ├── \_chapter{-}title.qmd}
\NormalTok{│   ├── figs/}
\NormalTok{│   ├── images/}
\NormalTok{│   ├── notebooks/}
\NormalTok{│   └── tables/}
\NormalTok{├── endmatter/}
\NormalTok{│   ├── appendix/}
\NormalTok{│   ├── bibliography/}
\NormalTok{│   └── oldfigures/}
\NormalTok{├── frontmatter/}
\NormalTok{│   ├── 01{-}dedication.txt}
\NormalTok{│   ├── 02{-}declaration.txt}
\NormalTok{│   ├── 03{-}abstract.txt}
\NormalTok{│   ├── 04{-}acknowledgements.txt}
\NormalTok{│   ├── before{-}body.tex}
\NormalTok{│   ├── figs/}
\NormalTok{│   ├── header.tex}
\NormalTok{│   ├── title.tex}
\NormalTok{│   └── toc.tex}
\end{Highlighting}
\end{Shaded}

\begin{enumerate}
\def\labelenumi{\arabic{enumi}.}
\setcounter{enumi}{3}
\tightlist
\item
  The \_quarto.yml specifies some variables and your document structure,
  make sure to declare them in there if you change relevent file names
  or paths, or create new files you want to include take note of the
  structure above and the .yml below. To get this chapter to print, I
  will have to add ``- chapter-03/\_chapter-title.qmd'' to the .yml file
  below. No need to worry about figures etc.
\end{enumerate}

\begin{Shaded}
\begin{Highlighting}[]
\FunctionTok{project}\KeywordTok{:}
\AttributeTok{  }\FunctionTok{type}\KeywordTok{:}\AttributeTok{ book}
\FunctionTok{book}\KeywordTok{:}
\AttributeTok{  }\FunctionTok{chapters}\KeywordTok{:}
\AttributeTok{    }\KeywordTok{{-}}\AttributeTok{ index.qmd}
\AttributeTok{    }\KeywordTok{{-}}\AttributeTok{ chapter{-}00/\_introduction.qmd}
\AttributeTok{    }\KeywordTok{{-}}\AttributeTok{ chapter{-}01/\_literature{-}review.qmd}
\AttributeTok{    }\KeywordTok{{-}}\AttributeTok{ chapter{-}02/\_materials\&methods.qmd}
\AttributeTok{    }\KeywordTok{{-}}\AttributeTok{ endmatter/bibliography/\_references.qmd}
\AttributeTok{  }\FunctionTok{appendices}\KeywordTok{:}
\AttributeTok{    }\KeywordTok{{-}}\AttributeTok{ endmatter/appendices/appendix{-}a.qmd}

\FunctionTok{title}\KeywordTok{:}\AttributeTok{ Deep models for shallow waters}
\FunctionTok{author}\KeywordTok{:}\AttributeTok{ James Harding}
\FunctionTok{college}\KeywordTok{:}\AttributeTok{ School for Infrastructure and Environment}
\FunctionTok{degree}\KeywordTok{:}\AttributeTok{ Doctor of Philosophy}
\FunctionTok{year}\KeywordTok{:}\AttributeTok{ }\DecValTok{2023}
\FunctionTok{format}\KeywordTok{:}
\AttributeTok{  }\FunctionTok{pdf}\KeywordTok{:}
\AttributeTok{    }\FunctionTok{keep{-}tex}\KeywordTok{:}\AttributeTok{ }\CharTok{false}\AttributeTok{ }
\AttributeTok{    }\FunctionTok{documentclass}\KeywordTok{:}\AttributeTok{ uoe{-}thesis{-}template}
\AttributeTok{    }\FunctionTok{classoption}\KeywordTok{:}\AttributeTok{ }\KeywordTok{[}\AttributeTok{twoside}\KeywordTok{]}
\AttributeTok{    }\FunctionTok{papersize}\KeywordTok{:}\AttributeTok{ a4}
\AttributeTok{    }\FunctionTok{template{-}partials}\KeywordTok{:}
\AttributeTok{    }\KeywordTok{{-}}\AttributeTok{ frontmatter/title.tex}
\AttributeTok{    }\KeywordTok{{-}}\AttributeTok{ frontmatter/before{-}body.tex}
\AttributeTok{    }\KeywordTok{{-}}\AttributeTok{ frontmatter/toc.tex}
\AttributeTok{    }\FunctionTok{include{-}in{-}header}\KeywordTok{:}
\AttributeTok{    }\KeywordTok{{-}}\AttributeTok{ frontmatter/header.tex}
\FunctionTok{supervisors}\KeywordTok{: }\CharTok{|}
\NormalTok{  Dr Dre.}
\FunctionTok{wordcount}\KeywordTok{: }\CharTok{|}
\NormalTok{  tbc}
\FunctionTok{keywords}\KeywordTok{:}\AttributeTok{ }\KeywordTok{[}\AttributeTok{template}\KeywordTok{,}\AttributeTok{ demo}\KeywordTok{]}
\FunctionTok{bibliography}\KeywordTok{:}\AttributeTok{ endmatter/bibliography/wq.bib}
\end{Highlighting}
\end{Shaded}

\bookmarksetup{startatroot}

\chapter*{References}\label{references}
\addcontentsline{toc}{chapter}{References}

\markboth{References}{References}

\phantomsection\label{refs}
\begin{CSLReferences}{1}{0}
\bibitem[\citeproctext]{ref-ca1}
EU-WFD. n.d. {``{Directive 2000/60/EC of the European Parliament and of
the Council of 23 October 2000 establishing a framework for Community
action in the field of water policy}.''}

\end{CSLReferences}

\cleardoublepage
\phantomsection
\addcontentsline{toc}{part}{Appendices}
\appendix

\chapter{Appendix A}\label{appendix-a}

You can write appendixes in the same way as chapters. Just add the class
\texttt{appendix} to the header. Be sure to add it as an appendix in
\texttt{\_quarto.yml} as well.




\end{document}
